%ARQUIVO DE PREAMBULO DA TESE - PACOTES E CONFIGURA\c{C}\~{O}ES

\documentclass[
	% -- op\c{c}\~{o}es da classe memoir --
	12pt,				% tamanho da fonte
	openright,			% cap\'{\i}tulos come\c{c}am em p\'{a}g \'{\i}mpar (insere p\'{a}gina vazia caso preciso)
	twoside,			% para impress\~{a}o em verso e anverso. Oposto a oneside
	letterpaper,		% tamanho do papel.
	% -- op\c{c}\~{o}es da classe abntex2 --
	%chapter=TITLE,		% t\'{\i}tulos de cap\'{\i}tulos convertidos em letras mai\'{u}sculas
	%section=TITLE,		% t\'{\i}tulos de se\c{c}\~{o}es convertidos em letras mai\'{u}sculas
	%subsection=TITLE,	% t\'{\i}tulos de subse\c{c}\~{o}es convertidos em letras mai\'{u}sculas
	%subsubsection=TITLE,% t\'{\i}tulos de subsubse\c{c}\~{o}es convertidos em letras mai\'{u}sculas
	% -- op\c{c}\~{o}es do pacote babel --
	brazil,			% idioma adicional para hifeniza\c{c}\~{a}o
	%french,			% idioma adicional para hifeniza\c{c}\~{a}o
	%spanish,			% idioma adicional para hifeniza\c{c}\~{a}o
	english,				% o \'{u}ltimo idioma \'{e} o principal do documento
	sumario=tradicional,
	]{abntex2}

\raggedbottom
% ---
% PACOTES
% ---
% ---
% Pacotes fundamentais
% ---

\usepackage{cmap}				% Mapear caracteres especiais no PDF
\usepackage{lmodern}			% Usa a fonte Latin Modern			
\usepackage[T1]{fontenc}		% Selecao de codigos de fonte.
\usepackage{lastpage}			% Usado pela Ficha catalogr\'{a}fica
\usepackage[table,xcdraw]{xcolor}				% Controle das cores
\usepackage[pdftex]{graphicx}	% Inclus\~{a}o de gr\'{a}ficos
\usepackage{epstopdf}           % Pacote que converte as figuras em eps para pdf
\usepackage{lipsum}             % Pacote que gera texto dummy
\usepackage{blindtext}          % Pacote que gera texto dummy
\usepackage{tikz}
%\usepackage{chronosys}
\usepackage{longtable}
\usepackage{nomencl}
\usepackage{amsmath}
\usepackage{bbm}
\usepackage[chapter]{algorithm}
\usepackage{algorithmic}
\usepackage{rotating}
\usepackage{pdfpages}
% ---
% Basic packages
\usepackage{lmodern}			% Usa a fonte Latin Modern			
\usepackage[T1]{fontenc}		% Selecao de codigos de fonte.
\usepackage[utf8]{inputenc}		% Codificacao do documento (conversão automática dos acentos)
\usepackage{lastpage}			% Usado pela Ficha catalográfica
\usepackage{indentfirst}		% Indenta o primeiro parágrafo de cada seção.
\usepackage{color}				% Controle das cores
\usepackage{graphicx}			% Inclusão de gráficos
\usepackage{microtype} 			% para melhorias de justificação
\usepackage{epsfig,psfrag,amsmath,tabularx} % para figuras eps e simbolos matematicos



% Packages
\catcode`\_=12
\usepackage{titlesec}
\usepackage[framemethod=tikz]{mdframed}
\setcounter{secnumdepth}{4}
\usepackage{setspace}
\usepackage{float}
\usepackage{enumitem}
\usepackage{listings}
\usepackage{pgfplots}
\usepackage{caption}
\usepackage{subcaption}
\usepackage{hyperref}
\usepackage{array}
\usepackage{multirow}
\usepackage{needspace}
\usepackage{framed}
% packet to add acronyms .tex file.
\usepackage[acronym,shortcuts]{glossaries}

\lstset{ %
language=C,                % choose the language of the code
basicstyle=\footnotesize,       % the size of the fonts that are used for the code
numbers=left,                   % where to put the line-numbers
numberstyle=\footnotesize,      % the size of the fonts that are used for the line-numbers
stepnumber=1,                   % the step between two line-numbers. If it is 1 each line will be numbered
numbersep=5pt,                  % how far the line-numbers are from the code
backgroundcolor=\color{white},  % choose the background color. You must add \usepackage{color}
showspaces=false,               % show spaces adding particular underscores
showstringspaces=false,         % underline spaces within strings
showtabs=false,                 % show tabs within strings adding particular underscores
frame=lines,           % adds a frame around the code
tabsize=2,          % sets default tabsize to 2 spaces
captionpos=b,           % sets the caption-position to bottom
breaklines=true,        % sets automatic line breaking
breakatwhitespace=false,    % sets if automatic breaks should only happen at whitespace
escapeinside={\%*}{*)}          % if you want to add a comment within your code
}

% ---
% Pacotes de cita\c{c}\~{o}es
% ---
\usepackage[english,hyperpageref]{backref}	 % Paginas com as cita\c{c}\~{o}es na bibl
\usepackage[alf,abnt-etal-cite=2,abnt-etal-list=0,abnt-etal-text=emph]{abntex2cite}	% Cita\c{c}\~{o}es padr\~{a}o ABNT

% ---
% Pacote de customiza\c{c}\~{a}o - Unicamp
% ---
\usepackage{unicamp}

\usepackage{footnote}

% ---
% CONFIGURA\c{C}\~{O}ES DE PACOTES
% ---

% ---
% Configura\c{c}\~{o}es do pacote backref
% Usado sem a op\c{c}\~{a}o hyperpageref de backref
\graphicspath{{./eps/}}
\DeclareGraphicsExtensions{.eps}

%customiza\c{c}\~{a}o do negrito em ambientes matem\'{a}ticos
\newcommand{\mb}[1]{\mathbf{#1}}
%customiza\c{c}\~{a}o de teoremas
\newtheorem{mydef}{Defini\c{c}\~{a}o}[chapter]
\newtheorem{lemm}{Lema}[chapter]
\newtheorem{theorem}{Teorema}[chapter]
\floatname{algorithm}{Pseudoc\'{o}digo}
\renewcommand{\listalgorithmname}{Lista de Pseudoc\'{o}digos}


\renewcommand{\backrefpagesname}{Cited on page(s):~}
% Texto padr\~{a}o antes do n\'{u}mero das p\'{a}ginas
\renewcommand{\backref}{}
% Define os textos da cita\c{c}\~{a}o
\renewcommand*{\backrefalt}[4]{
	\ifcase #1 %
		Not cited on the text.%
	\or
		Cited on page #2.%
	\else
		Cited #1 times on page(s) #2.%
	\fi}%
% ---


% ---
% Configura\c{c}\~{o}es de apar\^{e}ncia do PDF final

% alterando o aspecto da cor azul
\definecolor{blue}{RGB}{41,5,195}

% informa\c{c}\~{o}es do PDF
\makeatletter
\hypersetup{
     	%pagebackref=true,
		pdftitle={\@title},
		pdfauthor={\@author},
    	pdfsubject={\imprimirpreambulo},
	    pdfcreator={LaTeX with abnTeX2},
		pdfkeywords={abnt}{latex}{abntex}{abntex2}{trabalho acad\^{e}mico},
		hidelinks,					% desabilita as bordas dos links
		colorlinks=false,       	% false: boxed links; true: colored links
    	linkcolor=blue,          	% color of internal links
    	citecolor=blue,        		% color of links to bibliography
    	filecolor=magenta,      	% color of file links
		urlcolor=blue,
%		linkbordercolor={1 1 1},	% set to white
		bookmarksdepth=4
}
\makeatother
% ---

% ---
% Espa\c{c}amentos entre linhas e par\'{a}grafos
% ---

% O tamanho do par\'{a}grafo \'{e} dado por:
\setlength{\parindent}{1.3cm}

% Controle do espa\c{c}amento entre um par\'{a}grafo e outro:
\setlength{\parskip}{0.2cm}  % tente tamb\'{e}m \onelineskip

% ---
% Informacoes de dados para CAPA e FOLHA DE ROSTO
% ---
\titulo{Architecture and implementation of a Broadband Network Gateway using a programmable dataplane processor}
\autor{Juan Sebastian Mejia Vallejo}
\local{Campinas}
\data{2018}
\orientador{Prof. Dr. Christian Rodolfo Esteve Rothenberg}
\instituicao{%
    UNIVERSIDADE ESTADUAL DE CAMPINAS
    \par
    Faculdade de Engenharia El\'{e}trica e de Computa\c{c}\~{a}o	
    }
%\tipotrabalho{Tese (Doutorado)}
%% O preambulo deve conter o tipo do trabalho, o objetivo, o nome da institui\c{c}\~{a}o e a \'{a}rea de concentra\c{c}\~{a}o
%\preambulo{Tese apresentada \`{a} Faculdade de Engenharia El\'{e}trica e de Computa\c{c}\~{a}o da Universidade Estadual de Campinas como parte dos requisitos exigidos para a obten\c{c}\~{a}o do t\'{\i}tulo de Doutor em Engenharia El\'{e}trica, na \'{A}rea de Engenharia de Computa\c{c}\~{a}o.}
\tipotrabalho{Disserta\c{c}\~{a}o (Mestrado)}
\preambulo{Disserta\c{c}\~{a}o apresentada \`{a} Faculdade de Engenharia El\'{e}trica e de Computa\c{c}\~{a}o da Universidade Estadual de Campinas como parte dos requisitos exigidos para a obten\c{c}\~{a}o do t\'{\i}tulo de Mestre em Engenharia El\'{e}trica, na \'{A}rea de Engenharia de Computação.}
% --- 
